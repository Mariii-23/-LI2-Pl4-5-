Laboratórios de Informática $\vert$$\vert$ \mbox{[}M\+I\+E\+I\+NF\mbox{]} Universidade do Minho Turno\+: P\+L4 Grupo\+: 5 Jéssica Macedo Fernandes -\/ a93318 Mariana Dinis Rodrigues -\/ a93229 Vasco Oliveira Matos -\/ a93206

Para o início do trabalho seguimos os passos do \char`\"{}\+Guião aula 5\char`\"{} disponiblizado pelos docentes de L\+I2. Assim, criamos os vários ficheiros .c, \char`\"{}camada de dados\char`\"{}, \char`\"{}\+Interface\char`\"{} e \char`\"{}lógica do programa\char`\"{}, e os seus respetivos .h. Na parte da \char`\"{}camada de dados\char`\"{} está definida a estrutura de dados do programa, onde é definido o conceito de \char`\"{}\+C\+A\+S\+A\char`\"{}(cada local do jogo onde as peças se podem movimetar), a \char`\"{}\+C\+O\+R\+D\+E\+N\+A\+D\+A\char`\"{} (que serve para localizar cada peça na sua casa), entre outros dados essenciais. Na parte \char`\"{}\+Lógica do programa\char`\"{} estão presentes as funções que vão efetuar altreções no estado do jogo, nomeadamente na troca do estado das peças, e alterar a casa da peça em que estava, colocando uma peça branca na nova localização e uma preta na localização anterior(usando como auxiliar a função altera\+\_\+estado\+\_\+peca). Estas duas funções vão ser usadas na função \char`\"{}jogar\char`\"{} que aplica a jogada caso esta seja válida, ou retorna \char`\"{}\+Jogada Inválida\char`\"{} no caso de não ser. Na parte da \char`\"{}\+Interface\char`\"{} tempos presente uma função que desenha o estado do jogo \char`\"{}imprimindo\char`\"{} o tabuleiro por linha e por coluna(ao \char`\"{}imprimir\char`\"{} a linha, é \char`\"{}imprimido\char`\"{} cada elemento das colunas que intersetam essa linha). Existe ainda o ficheiro \char`\"{}main\char`\"{} do nosso programa onde é inicializado o jogo, e é chamada a funçao \char`\"{}interpertador\char`\"{} com o argumento \char`\"{}estado\char`\"{} afim de a camada de \char`\"{}\+Interface\char`\"{} imprimir o tabuleiro e trabalhar com as restantes funções do programa produzindo o jogo Rastos, posteriormente funcional.

Consoante o guião desta semana, o guião 6, foi necessário acrescentar ao nosso jogo as funções para verificar o jogador vencedor, que procura se a peça preta está na casa 1 ou 2, ou então, caso não seja possível esta movimentar-\/se, determina o vencedor. Implementamos também a função Q que consoante a decisão do jogador termina o jogo. O comado ler e gr recebem como argumento o ficheiro “jogo.\+txt” onde vão ler o tabuleiro e grava-\/lo, respectivamente. Criamos também o prompt com o objetivo de gravar o tabuleiro e as jogadas efetuadas, e o jogador que as efetuou. Todo o jogo foi documentado com o Doxygen seguindo o exemplo dado pelos docentes, e corrido o comando doxigen para gerar a documentação. 